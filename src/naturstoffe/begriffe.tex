
\section{Begriffe} \label{sec:begriffe}
\subsection{Isomerie}
Stoffe mit der selben Summenformel aber unterschiedlicher Strukturformel. (Andere räumliche Struktur)
\begin{enumerate}
    \item Konstitutionsisomere: \\
        Beispiel: 2-Methylpropan und Butan

    \item Stereoisomere \\
        Unterscheiden sich nur in der räumlichen Anordnung unterschieden
    \item Enantionmere \\
        Sind Stereoisomere; Spiegelbild aber nicht identisch
    \item Diastomere
        Sind Stereoisomiere; Keine Enantiomere -> nicht im Spiegelbild gleich

\end{enumerate}
\subsection{Asymmetrisches C-Atom}
Ein C-Atom bei dem alle "ärmchen" unterschiedlche Bindungen haben. \\
Beispiel:
\begin{enumerate}
    \item - H
    \item - OH
    \item - Methan
    \item - Ethan
\end{enumerate}

\subsection{Chiralität}


\subsection{Disacchararide}
Zwei Monosaccharide die über eine Sauerstoffverbindung beknüpft sind.
\subsubsection{Besonderheiten zum Bau der Saccharose:}
$\alpha$-D-Glucose und $\beta$-D Fructose (Glucopyronose) zu einer $\alpha 1 -\beta 2$ glycosidischen B. \\ $->$ $\alpha-\beta$-1,2-... \\
Saccharose muss gezeichnet werden können.

\subsubsection{Besonderheiten zum Bau der Maltose:}
Es gibt eine $\alpha$-Maltose + $\beta$-Maltose. \\
$\alpha$-1,4 glycosidische Verknünpfung einer $\alpha$-D Glucose mti einer $\alpha/\beta$-D Glucose

\subsubsection{Besonderheiten zum Bau der Cellobiose:}
$\beta$-D Glucose mit $\beta$-D Glucose \\
$\beta$-1,4-glycosidische Verknüpfung.

\subsubsection{Besonderheiten zum Bau der Lactose:}
D-Galactose mit D-Glucose \\
$\beta$-1,4 glycosidische Bindung

