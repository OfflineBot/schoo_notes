\section{pH-Wert} \label{sec:ph_wert}

Der pH Wert ist ein Maß für den sauren oder basischen Charakter eine wässrigen Lösung.
Er ist die Gegenzahl zum Logarithmus der Ammonium-Ionen $(H_3O^+)$ des Stoffes.
Um die stärke der jeweiligen Säure/Base zu bestimmen gibt es den \hyperref[sec:pks_pkb]{$pK_s$} oder \hyperref[sec:pks_pkb]{$pK_b$} Wert. \\
0: stark sauer \\
4: leicht sauer \\
7: neutral \\
10: leicht basich \\
14: stark basisch

\subsection{pKs/pKb Berechnung}
starke Säuren: \\
$pH = -lg([c(H_3O^+)]) = -lg([c_0(HAc)])$ \\
schwache Säuren: \\
$pH = \frac{1}{2} (pK_s - lg([c_0(HAc)]))$ \\
starke Basen: \\
$pOH = -lg([c(OH^-)])$ \textrightarrow\ $pH = 14 - pOH$ \\
schwache Basen: \\
$pOH = \frac{1}{2} (pK_s - lg([c_0(Ac)]))$ \\
