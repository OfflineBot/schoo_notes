\section{pKs/pKb Wert} \label{sec:pks_pkb}
Je niedriger der $pK_s$-Wert desto stärker die Säure. \\
Je niedriger der $pK_b$-Wert desto stärker die Base. \\
Der \hyperref[sec:ph_wert]{pH Wert} wird mithilfe des $pK_s$-$pK_b$-Wert berechnet. \\
$pK_s + pK_b = 14$ \\

\subsection{pKs/pKb Berechnen}
Nach dem \hyperref[sec:mwg]{Massenwirkungsgesetz}: \\
$K_s = \frac{c(A^-)*c(H_3O^+)}{c(HA)}$ \\
$K_b = \frac{c(HB^-)*c(OH^-)}{c(B)}$ \\

\begin{itemize}
    \item $K_s$ = Säurenkonstante
    \item $H_3O^+$ = Hydroxidion
    \item $A^-$ = konjugierte Base
    \item $HA$ = schwache Säure
\end{itemize}
\
$pK_s = -lg(K_s)$ \\
$pK_b = -lg(K_b)$

