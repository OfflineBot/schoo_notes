
\section{Endotherm und Exotherm}\label{sec:endo_exotherm}
Mithilfe der \hyperref[sec:aktivierungsenergie]{Aktivierungsenergie} und der Differenz der Energien zwischen Edukt und Produkt,
lässt sich sagen, ob eine Reaktion endotherm oder exotherm ist.
Wenn die \hyperref[sec:aktivierungsenergie]{Aktivierungsenergie} kleiner ist als die Differenz, dann ist die Reaktion exotherm.
Ist die Aktivierungsenergie höher als die Differenz, dann ist die Reatkion endotherm. \\ \\
Endotherm:\\
Das System nimmt Energie aus der Umgebung auf. 
Während der Reaktion nimmt die Temperatur ab.
Ein Beispiel wäre das verdampfen von Wasser. \\ \\
Exotherm: \\
Das System gibt Energie an das System in Form von wärme ab.
Ein Beispiel wäre das verbrennen von Holz.

