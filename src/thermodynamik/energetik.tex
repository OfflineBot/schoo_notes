
\section{Energetik} \label{sec:energetik}


\subsection{Chemische Reaktionen und Energie}
Alle Stoffe sind mit Energieumwandlungen gekoppelt.
Chemische Energie kann umgewandelt werden in:
\begin{itemize}
    \item thermische Energie
    \item kinetische Enerige (Mechanische Energie)
    \item elektrische Energie
    \item Strahlungsenergie
\end{itemize}


\subsection{System und Umgebung}
Will man Energieumsätze bei chemischen Reatkionen betrachten, muss man festlegen in welchem räumlichen Bereich die Untersuchung statt finden soll.
Ein solch abgegrenzter Raum nennt man \underline{System}. Der Bereich der übrig bleibt wird \underline{Umgebung} genannt. \\
Arten:
\begin{itemize}
    \item Offenes System: 
        Energie + Stoffaustausch (Reagenzglas)
    \item Geschlossenes System:
        Energieaustausch (Reagenzglas + Stopfen)
    \item Isoliertes System: 
        kein Austausch  (Reagenzglas + Stofen + Styropor)
\end{itemize}


\subsection{Die innere Energie}
Die Summe aller Energie die in einem System steckt:
\begin{itemize}
    \item chemische Energie
    \item thermische Energie
    \item kinetische Energie
\end{itemize}
Einheit in Joule [J].\\
Die innere Energie ist eine extensive Größe die von der Stoffmenge abhänging ist. Daher ist der Unterschied $\Delta$U messbar.


\subsection{Energieerhaltungssatz}
1. Hauptsatz der Thermodynamik: \\
Die Gesamtenergie eines Systems und seine Umgebung kann weder zu, noch abnehmen. \\ \\
\hyperref[sec:endo_exotherm]{exotherme} Reaktion $Q_r$ \textless 0 (-E) \\
\hyperref[sec:endo_exotherm]{endotherme} Reaktion $Q_r$ \textgreater 0 (+E) \\ \\
Um $Q_r$ bestimmen zu können, begrenzt man die Umgebung künstlich auf einen definierten Raum.
Dieser muss isoliert sein.
Solche "Systeme" sind \hyperref[sec:kalorimeter]{Kalorimeter}. \\ 
Kann vernachlässigt werden: \\

\begin{itemize}
    \item Reaktionswärme die im geschlossenen System bleibt
    \item Wärmeverlust durch den Deckel des \hyperref[sec:kalorimeter]{Kalorimeters}
\end{itemize} 
\ \\
Je nach Reaktionstyp unterscheidet man:

\begin{itemize}
    \item Verbrennungswärme
    \item Lösungswärme
    \item Neutralisationswärme
    \item Bildungswärme
\end{itemize}
